In recent years, due to the emergence of next-generation sequencing technology (NGS), we can sequence a large amount of DNA at the same time, greatly reducing the cost of sequencing, and many related studies have also been greatly expanded. The variant calling in NGS data is almost the key step for all downstream analysis and interpretation processes, but there are still many problems and challenges for accurate variant calling. Therefore, we have previously proposed an explicit alternative gene likelihood evaluator (EAGLE), which deals with the inherent uncertainty in the process of variant calling by using the display probability model. Explicitly evaluates how well sequencing data fit the alternative genome sequence implied by a putative variant. But whether it is the EAGLE model or the general variant calling method, it depends more or less on the result of comparison the read sequence to the reference genome sequence. The alignment method is to compare the read sequence with the reference genome sequence, and determine their possible position on the reference genome sequence according to the similarity of the comparison.

Because the target reference genome sequence we compared is linear, contains only one sequence, the diversity of genome cannot be captured. So, there may be some errors in the alignment, which may cause our reads mapped to the wrong position or cannot be mapped. This phenomenon is called reference bias. Wrong sequence alignment will lead to false negative or false positive variant calls, which further affects the accuracy of our recognition of the variants. However, many studies have been carried out to eliminate the influence of reference bias. One of the methods is to improve the reference genome sequence to include the diversity of genome, so as to achieve better alignment results. In this paper, we propose a similar concept method and apply it to the EAGLE which we proposed earlier to explore whether the effect of the proposed method on reducing reference bias is feasible and its potential.

Our method can be divided into five parts. The first part is obtained the variant information through the variant file (VCF), and combining the reference genome sequence to establish the hypothetical sequence. The second part is to create an index structure for read sequence file (FASTQ) that can be quickly searched. The third part is to use Burrows-Wheeler Aligner (BWA) and the reading index structure we built in the previous part to find all reads in the read file that can match the hypothetical sequence we created. The fourth part is to check the pileup in the original alignment file (BAM). The pileup is all the read orders mapped to the same area in the alignment results, check whether the read sequence exists in the original pileup, and establish relevant alignment information for those read sequences that do not exist in the pileup. The fifth part is to add the reads and alignment information we find into the pileup to improve the calculation results of our eagle

In the experiment, we simulate the occurrence of reference bias, and verify whether we can find the missing reads affected by the reference bias by our method. The experimental results show that we can find most of the missing read orders, but there may be some restrictions in some region with too many repeat sequences. We also tested the dbSNP variant data, and basically, we can get similar results, but the effect on single nucleotide polymorphism variation has not achieved the ideal results. But generally speaking, based on the time we have increased and the results we can achieve, we think that our method has considerable potential in solving the impact of reference bias, but it still needs further verification and careful explanation to apply it to the actual situation.

\begin{flushleft}
\mbox{{\bf Keywords}:  NGS$B!"(Bvariant calling$B!"(Breference bias }
\end{flushleft}
