In recent years, due to the emergence of next-generation sequencing technology (NGS), we can sequence a large amount of DNA at the same time, greatly reducing the cost of sequencing, and expanding its applications.  Genome variant calling from NGS data is a key step for many applications in biology and medicine, but there are still many problems and challenges for accurate variant calling.  Most variant calling procedures start with the so called ``pile-up'', an alignment of reads to the reference genome around the position of a putative variant.  However this alignment cannot be expected to be certain; because the human genome is repetitive (many distinct regions are be similar to each other) and the reference genome sequence differs from any individual.  Moreover alignment only to the reference genome naturally tends to favor the reference sequence over others (so called reference bias).

Reference bias is a widely recognized problem in variant calling.  One research direction is to represent the reference genome with a graph instead of a simple string to include some common variants.  In a different approach, we previously addressed reference bias by proposing a method (EAGLE: Explicit Alternative Gene Likelihood Evaluator) which locally aligns reads not only to the reference sequence, but also to putative variant sequences; using an explicit probability model to combine evidence from all possible local alignments.  The results were encouraging but fell short of completely eliminating reference bias, because that version of EAGLE still relies on read mapping to identify the reads which should be considered when evaluating a variant.

To completely eliminate reference bias in putative variant evaluation, any read similar to the variant sequence should be considered, even it does not match the reference sequence.  Na\"ively matching all reads to all variants would be far too slow, but in principle the search indexes used to accelerate standard read mapping could be adapted to search reads against a variant sequence query.  But one might ask 1) if this would be feasible in practice (in terms of computer memory and time) and 2) if it would really make a difference in the variant calling results.  Last year we conducted a feasibility study which suggested that constructing a search index on reads is feasible for whole exome sequencing and that searching such a read index can indeed find relevant reads not present in the ``pile-up''.  That study however was only a feasibility study and did not integrate the read index with the EAGLE variant likelihood computation.

Here we explore the read index idea more thoroughly and integrate it into the EAGLE software.
Our method can be divided into two preprocessing steps and then four steps for each putative variant.  In the first pre-processing step we align the reads to the reference genome.  In the second pre-processing step, we create a Burrows-Wheeler Aligner (BWA) index structure for the read sequence data (FASTQ format file) that can be quickly queried.  Then for each variant in a list of putative variants (VCF format file), we first edit the reference genome to include the putative variant, then use that to query the read index structure, merge the matching reads with any reads mapping to the reference genome near the putative variant genome position, and finally apply the EAGLE probabilistic model computation to the merged set of reads to obtain an unbiased likelihood ratio between variant and reference sequence for that position.

We performed some experiments to evaluate the method.  In one experiment we use real sequence read data but doctor the reference genome to produce a false reference genome sequence, thus simulating a situation in which we know the position of differences between individual and reference.  Then we map reads to this false genome sequence to obtain a corresponding false pile-up.   We tested if our read index can find reads supporting the actual genome sequence, but missing in the false pile-up.  The experimental results show that we can find most of the missing reads, but have some difficulty in regions with repetitive sequences.  We also tested real putative variants from the dbSNP dataset, observing qualitatively similar results.  We show that for putative indel variants, the extra reads found by the read index have a large effect on the EAGLE likelihood ratio (and in the correct direction in cases where we know what the answer should be).  The likelihood ratio of Single Nucleotide Variants (SNV)s, on the other hand is not greatly affected, a reasonable outcome since standard read mapping can tolerate single mismatchs, largely obviating the need to use a special read index.  More comprehensive benchmarking is left for future work.


\begin{flushleft}
\mbox{{\bf Keywords}:  NGS、variant calling、reference bias }
\end{flushleft}
