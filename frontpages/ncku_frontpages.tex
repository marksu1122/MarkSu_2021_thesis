%% // nckuee.sty 定義 // cbj

% 產生論文封面
\nckuEEtitlepage
% 產生口試委員會簽名單
%\nckuEEoralpage
% 產生口試委員簽名單(en)
%\nckuEEenoralpage

%\newpage
%\setcounter{page}{1}
%\pagenumbering{roman}

%%%%%%%%%%%%%%%%%%%%%%%%%%%%%%%
%       封面內頁
%%%%%%%%%%%%%%%%%%%%%%%%%%%%%%%
% % unmark to add inner cover
%\newpage
%\thispagestyle{empty}
%\thispagestyle{EmptyWaterMarkPage}
%\nckuEEtitlepage


%%%%%%%%%%%%%%%%%%%%%%%%%%%%%%%
%       中文摘要
%%%%%%%%%%%%%%%%%%%%%%%%%%%%%%%

% 可以利用如下自定義的command (定義在nckuee.sty)
% ======
%\begin{zhAbstract}  %中文摘要
%近年來由於次世代定序技術(NGS)的出現,使得我們可以同時對大量DNA進行定序,大幅降低定序成本,許多相關的研究也跟著大幅擴展,NGS數據中變異調用幾乎是所有下游分析和解釋過程都依賴的關鍵步驟,但是對於準確基因變異體偵測仍然有許多問題及挑戰。因此我們先前提出過一個工具Explicit Alternative Genome Likelihood Evaluator(EAGLE),透過使用顯示概率模型,處理變異調用過程中固有的不確定性。明確評估測序數據與推定變異所隱含的替代基因組序列的匹配程度。但是不論是EAGLE的模型抑或是一般的變異調用方法,或多或少都仰賴將讀序映射到參考基因組序列上的結果。而映射的方法是透過將讀序與參考基因組序列比對,根據比對的相似程度決定他們可能來自參考基因組序列上的位置。

由於我們比對的目標參考基因組序列是線性的,僅包含單個序列,因此無法捕獲基因組的多樣性。所以比對會出現可能一定程度的誤差,造成我們讀序可能被映射到錯誤的位置或是無法映射,這種現象我們稱之參考偏差。錯誤的讀序映射會導致假陰性或假陽性變異調用,進一步影響到我們對於辨認基因體變異的準確性。而先前已經有許多研究討論消除參考偏差所帶來的影響,其中一種方法就是透過改進參考基因組序列,使其能夠包含基因組的多樣性,從而達到更好的映射結果。本文中我們提出一個類似概念的方法,並應用於我們先前提出的工具-EAGLE上,探討我們提出的方法對於降低參考偏差帶來的影響是否可行以及其潛力。

我們的方法可以分成五個部分,第一個部分受透過基因突變資料(VCF)中取得變異點資訊,結合參考基因組序列建立假設序列。第二個部分是為讀序檔案(FASTQ)建立一個能夠快速搜尋的索引結構。第三個部分是透過Burrows-Wheeler Aligner(BWA)以及我們上一部份建立的讀序索引結構在讀序檔案中找出所有能夠與我們建立的假設序列匹配的讀序。第四個部分則是檢查原本的映射結果檔案(BAM)中的堆積序列,堆積序列是在映射結果中被映射到相同區域的所有讀序,檢查是否存在原有的堆積序列中,對那些不存在堆積序列中的讀序建立相關的比對資訊。第五部分則是將我們找到的讀序以及比對資訊加入堆積序列中,改進我們EAGLE的計算結果。

在實驗中,我們模擬參考偏差發生的情景,透過我們的方法驗證是否能找到那些受到參考偏差的影響而遺失的讀序,結果顯示我們能夠找到當中大部分被遺失的讀序,但是在一些重複序列過多的區域可能有所限制。我們同時也在 dbSNP的變異資料及上進行測試,基本上也能夠取得相似的結果,惟在單核苷酸多態性的變異上效果沒有達到理想的成效。但大體而言,綜合我們所增加的時間以及成效,我們認為我們的方在在解決參考偏差的影響是相當有潛力的,但是應用到實際情況還是需要更進一步的驗證以及謹慎的解釋。

\begin{flushleft}
\mbox{{\bf 關鍵字}: 次世代定序、變異點偵測、參考偏差 }
\end{flushleft}
 % // 可以引入front_cabstract.tex檔案或在此編輯 // cbj
%\end{zhAbstract}

% ...等
% ======

% 在此直接定義如下
%%%%%%%%%%%%%%%%
%
\newpage
% // HongJhe 頁碼起始
\setcounter{page}{1}
\pagenumbering{roman}
% create an entry in table of contents for 中文摘要
\phantomsection % for hyperref to register this
\addcontentsline{toc}{chapter}{\nameCabstract}
% aligned to the center of the page
\begin{center}
% font size (relative to 12 pt):
% \large (14pt) < \Large (18pt) < \LARGE (20pt) < \huge (24pt)< \Huge (24 pt)
% Set the line spacing to single for the names (to compress the lines)
\renewcommand{\baselinestretch}{1}   %行距 1 倍
% it needs a font size changing command to be effective
\LARGE{\zhTitle}\\  %中文題目
\vspace{0.83cm}
% \makebox is a text box with specified width;
% option s: stretch
% use \makebox to make sure
% each text field occupies the same width
%\makebox[1.5cm][c]{\large{學生:}}
\hspace{0.5in}
\renewcommand{\thefootnote}{\fnsymbol{footnote}}
\makebox[3.5cm][l]{\large{\authorZhName\footnote[1]{}}}\footnotetext[1]{{學生}} % 學生中文姓名
%\hfill
%
%\makebox[3cm][c]{\large{指導教授:}}
\makebox[3.5cm][l]{\large{\advisorZhName\footnote[2]{}}}\footnotetext[2]{{指導教授}} \\ %指導教授中文姓名
%
\vspace{0.42cm}
%
\large{\zhUniv}\large{\zhDepartmentName}\\ %校名系所名
\vspace{0.83cm}
%\vfill
\makebox[2.7cm][c]{\large{摘要}}
\end{center}
% Resume the line spacing to the desired setting
\renewcommand{\baselinestretch}{\mybaselinestretch}   %恢復原設定
%it needs a font size changing command to be effective
% restore the font size to normal
\normalsize
%%%%%%%%%%%%%
\par  % 摘要首段空格 by SianJhe
近年來由於次世代定序技術(NGS)的出現,使得我們可以同時對大量DNA進行定序,大幅降低定序成本,許多相關的研究也跟著大幅擴展,NGS數據中變異調用幾乎是所有下游分析和解釋過程都依賴的關鍵步驟,但是對於準確基因變異體偵測仍然有許多問題及挑戰。因此我們先前提出過一個工具Explicit Alternative Genome Likelihood Evaluator(EAGLE),透過使用顯示概率模型,處理變異調用過程中固有的不確定性。明確評估測序數據與推定變異所隱含的替代基因組序列的匹配程度。但是不論是EAGLE的模型抑或是一般的變異調用方法,或多或少都仰賴將讀序映射到參考基因組序列上的結果。而映射的方法是透過將讀序與參考基因組序列比對,根據比對的相似程度決定他們可能來自參考基因組序列上的位置。

由於我們比對的目標參考基因組序列是線性的,僅包含單個序列,因此無法捕獲基因組的多樣性。所以比對會出現可能一定程度的誤差,造成我們讀序可能被映射到錯誤的位置或是無法映射,這種現象我們稱之參考偏差。錯誤的讀序映射會導致假陰性或假陽性變異調用,進一步影響到我們對於辨認基因體變異的準確性。而先前已經有許多研究討論消除參考偏差所帶來的影響,其中一種方法就是透過改進參考基因組序列,使其能夠包含基因組的多樣性,從而達到更好的映射結果。本文中我們提出一個類似概念的方法,並應用於我們先前提出的工具-EAGLE上,探討我們提出的方法對於降低參考偏差帶來的影響是否可行以及其潛力。

我們的方法可以分成五個部分,第一個部分受透過基因突變資料(VCF)中取得變異點資訊,結合參考基因組序列建立假設序列。第二個部分是為讀序檔案(FASTQ)建立一個能夠快速搜尋的索引結構。第三個部分是透過Burrows-Wheeler Aligner(BWA)以及我們上一部份建立的讀序索引結構在讀序檔案中找出所有能夠與我們建立的假設序列匹配的讀序。第四個部分則是檢查原本的映射結果檔案(BAM)中的堆積序列,堆積序列是在映射結果中被映射到相同區域的所有讀序,檢查是否存在原有的堆積序列中,對那些不存在堆積序列中的讀序建立相關的比對資訊。第五部分則是將我們找到的讀序以及比對資訊加入堆積序列中,改進我們EAGLE的計算結果。

在實驗中,我們模擬參考偏差發生的情景,透過我們的方法驗證是否能找到那些受到參考偏差的影響而遺失的讀序,結果顯示我們能夠找到當中大部分被遺失的讀序,但是在一些重複序列過多的區域可能有所限制。我們同時也在 dbSNP的變異資料及上進行測試,基本上也能夠取得相似的結果,惟在單核苷酸多態性的變異上效果沒有達到理想的成效。但大體而言,綜合我們所增加的時間以及成效,我們認為我們的方在在解決參考偏差的影響是相當有潛力的,但是應用到實際情況還是需要更進一步的驗證以及謹慎的解釋。

\begin{flushleft}
\mbox{{\bf 關鍵字}: 次世代定序、變異點偵測、參考偏差 }
\end{flushleft}
 % // 可以引入front_eabstract.tex檔案或在此編輯 // cbj



%%%%%%%%%%%%%%%%%%%%%%%%%%%%%%%
%       英文摘要
%%%%%%%%%%%%%%%%%%%%%%%%%%%%%%%
%
%[method 1]

% 可以利用如下自定義的command (定義在nckuee.sty)
% ======
%\begin{enAbstract}  %英文摘要
%In recent years, due to the emergence of next-generation sequencing technology (NGS), we can sequence a large amount of DNA at the same time, greatly reducing the cost of sequencing, and many related studies have also been greatly expanded. The variant calling in NGS data is almost the key step for all downstream analysis and interpretation processes, but there are still many problems and challenges for accurate variant calling. Therefore, we have previously proposed an explicit alternative gene likelihood evaluator (EAGLE), which deals with the inherent uncertainty in the process of variant calling by using the display probability model. Explicitly evaluates how well sequencing data fit the alternative genome sequence implied by a putative variant. But whether it is the EAGLE model or the general variant calling method, it depends more or less on the result of comparison the read sequence to the reference genome sequence. The alignment method is to compare the read sequence with the reference genome sequence, and determine their possible position on the reference genome sequence according to the similarity of the comparison.

Because the target reference genome sequence we compared is linear, contains only one sequence, the diversity of genome cannot be captured. So, there may be some errors in the alignment, which may cause our reads mapped to the wrong position or cannot be mapped. This phenomenon is called reference bias. Wrong sequence alignment will lead to false negative or false positive variant calls, which further affects the accuracy of our recognition of the variants. However, many studies have been carried out to eliminate the influence of reference bias. One of the methods is to improve the reference genome sequence to include the diversity of genome, so as to achieve better alignment results. In this paper, we propose a similar concept method and apply it to the EAGLE which we proposed earlier to explore whether the effect of the proposed method on reducing reference bias is feasible and its potential.

Our method can be divided into five parts. The first part is obtained the variant information through the variant file (VCF), and combining the reference genome sequence to establish the hypothetical sequence. The second part is to create an index structure for read sequence file (FASTQ) that can be quickly searched. The third part is to use Burrows-Wheeler Aligner (BWA) and the reading index structure we built in the previous part to find all reads in the read file that can match the hypothetical sequence we created. The fourth part is to check the pileup in the original alignment file (BAM). The pileup is all the read orders mapped to the same area in the alignment results, check whether the read sequence exists in the original pileup, and establish relevant alignment information for those read sequences that do not exist in the pileup. The fifth part is to add the reads and alignment information we find into the pileup to improve the calculation results of our eagle

In the experiment, we simulate the occurrence of reference bias, and verify whether we can find the missing reads affected by the reference bias by our method. The experimental results show that we can find most of the missing read orders, but there may be some restrictions in some region with too many repeat sequences. We also tested the dbSNP variant data, and basically, we can get similar results, but the effect on single nucleotide polymorphism variation has not achieved the ideal results. But generally speaking, based on the time we have increased and the results we can achieve, we think that our method has considerable potential in solving the impact of reference bias, but it still needs further verification and careful explanation to apply it to the actual situation.


\begin{flushleft}
\mbox{{\bf Keywords}:  NGS、variant calling、reference bias }
\end{flushleft} % // 可以引入front_eabstract.tex檔案或在此編輯 // cbj
%\end{enAbstract}

%[method 2]
\newpage
% create an entry in table of contents for 英文摘要
\phantomsection % for hyperref to register this
\addcontentsline{toc}{chapter}{\nameEabstract} % // HongJhe marked

% aligned to the center of the page
\begin{center}
% font size:
% \large (14pt) < \Large (18pt) < \LARGE (20pt) < \huge (24pt)< \Huge (24 pt)
% Set the line spacing to single for the names (to compress the lines)
\renewcommand{\baselinestretch}{1}   %行距 1 倍
%\large % it needs a font size changing command to be effective
\LARGE{\enTitle}\\  %英文題目
\vspace{0.83cm}
% \makebox is a text box with specified width;
% option s: stretch
% use \makebox to make sure
% each text field occupies the same width
%\makebox[2cm][s]{\large{Student: }}
\hspace{0.45in}
\renewcommand{\thefootnote}{\fnsymbol{footnote}}
\makebox[5cm][l]{\large{\authorEnName\footnote[1]{}}}\footnotetext[1]{{Student}} % 學生英文姓名
%\hfill
%
%\makebox[2cm][s]{\large{Advisor: }}
\makebox[5cm][l]{\large{\advisorEnName\footnote[2]{}}}\footnotetext[2]{{Advisor}} \\ %教授英文姓名
%
\vspace{0.42cm}
\large{\enDepartmentName}\\ %英文系所全名
%
\large{\enUniv}\\  %英文校名
\vspace{0.83cm}
%\vfill
%
\large{\nameEabstractc}\\
%\vspace{0.5cm}
\end{center}

% Resume the line spacing the desired setting
\renewcommand{\baselinestretch}{\mybaselinestretch}   %恢復原設定
%\large %it needs a font size changing command to be effective
% restore the font size to normal
\normalsize
%%%%%%%%%%%%%
In recent years, due to the emergence of next-generation sequencing technology (NGS), we can sequence a large amount of DNA at the same time, greatly reducing the cost of sequencing, and many related studies have also been greatly expanded. The variant calling in NGS data is almost the key step for all downstream analysis and interpretation processes, but there are still many problems and challenges for accurate variant calling. Therefore, we have previously proposed an explicit alternative gene likelihood evaluator (EAGLE), which deals with the inherent uncertainty in the process of variant calling by using the display probability model. Explicitly evaluates how well sequencing data fit the alternative genome sequence implied by a putative variant. But whether it is the EAGLE model or the general variant calling method, it depends more or less on the result of comparison the read sequence to the reference genome sequence. The alignment method is to compare the read sequence with the reference genome sequence, and determine their possible position on the reference genome sequence according to the similarity of the comparison.

Because the target reference genome sequence we compared is linear, contains only one sequence, the diversity of genome cannot be captured. So, there may be some errors in the alignment, which may cause our reads mapped to the wrong position or cannot be mapped. This phenomenon is called reference bias. Wrong sequence alignment will lead to false negative or false positive variant calls, which further affects the accuracy of our recognition of the variants. However, many studies have been carried out to eliminate the influence of reference bias. One of the methods is to improve the reference genome sequence to include the diversity of genome, so as to achieve better alignment results. In this paper, we propose a similar concept method and apply it to the EAGLE which we proposed earlier to explore whether the effect of the proposed method on reducing reference bias is feasible and its potential.

Our method can be divided into five parts. The first part is obtained the variant information through the variant file (VCF), and combining the reference genome sequence to establish the hypothetical sequence. The second part is to create an index structure for read sequence file (FASTQ) that can be quickly searched. The third part is to use Burrows-Wheeler Aligner (BWA) and the reading index structure we built in the previous part to find all reads in the read file that can match the hypothetical sequence we created. The fourth part is to check the pileup in the original alignment file (BAM). The pileup is all the read orders mapped to the same area in the alignment results, check whether the read sequence exists in the original pileup, and establish relevant alignment information for those read sequences that do not exist in the pileup. The fifth part is to add the reads and alignment information we find into the pileup to improve the calculation results of our eagle

In the experiment, we simulate the occurrence of reference bias, and verify whether we can find the missing reads affected by the reference bias by our method. The experimental results show that we can find most of the missing read orders, but there may be some restrictions in some region with too many repeat sequences. We also tested the dbSNP variant data, and basically, we can get similar results, but the effect on single nucleotide polymorphism variation has not achieved the ideal results. But generally speaking, based on the time we have increased and the results we can achieve, we think that our method has considerable potential in solving the impact of reference bias, but it still needs further verification and careful explanation to apply it to the actual situation.


\begin{flushleft}
\mbox{{\bf Keywords}:  NGS、variant calling、reference bias }
\end{flushleft} % // 可以引入front_eabstract.tex檔案或在此編輯 // cbj


%%%%%%%%%%%%%%%%%%%%%%%%%%%%%%%
%       誌謝
%%%%%%%%%%%%%%%%%%%%%%%%%%%%%%%
%
% Acknowledgment
\newpage
\phantomsection % for hyperref to register this
%\addcontentsline{toc}{chapter}{\nameAcknc}

\begin{zhAckn}  %誌謝
Add your acknowledgements here.

\begin{flushright}
\mbox{Wen Hong Su}
\end{flushright} % // 可以引入front_ackn.tex檔案或在此編輯 // cbj
\end{zhAckn}

%\chapter*{\nameAckn} %\makebox{} is fragile; need protect
%Add your acknowledgements here.

\begin{flushright}
\mbox{Wen Hong Su}
\end{flushright} % // 可以引入my_ackn.tex檔案或在此編輯 // cbj
%%testjsjtoejiojsoijtoijos

%%%%%%%%%%%%%%%%%%%%%%%%%%%%%%%
%       目錄
%%%%%%%%%%%%%%%%%%%%%%%%%%%%%%%
%
% Table of contents
\newpage
\renewcommand{\contentsname}{\nameToc}
%\makebox{} is fragile; need protect
\phantomsection % for hyperref to register this
\addcontentsline{toc}{chapter}{\nameTocc}
\tableofcontents

%%%%%%%%%%%%%%%%%%%%%%%%%%%%%%%
%       表目錄
%%%%%%%%%%%%%%%%%%%%%%%%%%%%%%%
%
% List of Tables
\newpage
\renewcommand{\listtablename}{\nameLot}
%\makebox{} is fragile; need protect
\phantomsection % for hyperref to register this
\addcontentsline{toc}{chapter}{\nameLotc}
\listoftables

%%%%%%%%%%%%%%%%%%%%%%%%%%%%%%%
%       圖目錄
%%%%%%%%%%%%%%%%%%%%%%%%%%%%%%%
%
% List of Figures
\newpage
\renewcommand{\listfigurename}{\nameTof}
%\makebox{} is fragile; need protect
\phantomsection % for hyperref to register this
\addcontentsline{toc}{chapter}{\nameTofc}
\listoffigures
%%%%%%%%%%%%%%%%%%%%%%%%%%%%%%%
%       符號說明
%%%%%%%%%%%%%%%%%%%%%%%%%%%%%%%
%
% Symbol list
% define new environment, based on standard description environment
% adapted from p.60~64, <<The LaTeX Companion>>, 1994, ISBN 0-201-54199-8

%\newcommand{\SymEntryLabel}[1]%
%  {\makebox[3cm][l]{#1}}
%%
%\newenvironment{SymEntry}
%   {\begin{list}{}%
%       {\renewcommand{\makelabel}{\SymEntryLabel}%
%        \setlength{\labelwidth}{3cm}%
%        \setlength{\leftmargin}{\labelwidth}%
%        }%
%   }%
%   {\end{list}}
%%%
%\newpage
%\chapter*{\nameSlist} %\makebox{} is fragile; need protect
%\phantomsection % for hyperref to register this
%\addcontentsline{toc}{chapter}{\nameSlistc}
%%
% this file is encoded in utf-8
% v2.0 (Apr. 5, 2009)
%  各符號以 \item[] 包住,然後接著寫說明
% 如果符號是數學符號,應以數學模式$$表示,以取得正確的字體
% 如果符號本身帶有方括號,則此符號可以用大括號 {} 包住保護
\begin{SymEntry}

\item[OLED]
Organic Light Emitting Diode

\item[$E$]
energy

\item[$e$]
the absolute value of the electron charge, $1.60\times10^{-19}\,\text{C}$
 
\item[$\mathscr{E}$]
electric field strength (V/cm)

\item[{$A[i,j]$}]
the  element of the matrix $A$ at $i$-th row, $j$-th column\\
矩陣 $A$ 的第 $i$ 列,第 $j$ 行的元素

\end{SymEntry}

\newpage
\setcounter{page}{1}
\pagenumbering{arabic}
